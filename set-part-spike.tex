\documentclass{beamer}
\mode<presentation>{\usetheme{Warsaw} \setbeamercovered{transparent}}


\usepackage[english]{babel}
\usepackage[latin1]{inputenc}
\usepackage{times}
\usepackage[T1]{fontenc}
\usepackage{multicol}

\title[SetPart - https://github.com/test-load-balancer/set-part]{SetPart}

\subtitle{genetic algoritm based set partitioning algoritm}

\author[Pavan, Janmejay]{Pavan KS\inst{1} \and Janmejay Singh\inst{2}}

\institute[ThoughtWorks Studios]{
  \inst{1}%
  mail: itspanzi@gmail.com\\
  blog: http://itspanzi.blogspot.com
  \and
  \inst{2}%
  mail: singh.janmejay@gmail.com\\
  blog: http://codehunk.wordpress.com }

\date[set-part]{Course Project Evaluation(CCE), 2010}

\subject{Introduction to the problem}

\AtBeginSubsection[]{ \begin{frame}<beamer>{Outline} \tableofcontents[currentsection,currentsubsection] \end{frame} }

\begin{document}

\begin{frame}
  \titlepage
\end{frame}

\begin{frame}{Outline}
  \tableofcontents
  % You might wish to add the option [pausesections]
\end{frame}

\section{Motivation}

\subsection{Description of the problem}

\begin{frame}{Informal definition}
  \begin{center}
    A partition of a set \text{X} is a division of \text{X} into non-overlapping and non-empty \text{"parts"} or \text{"blocks"} or \text{"cells"} that cover all of \text{X}. More formally, these \text{"cells"} are both \emph{collectively exhaustive} and \emph{mutually exclusive} with respect to the \text{set being partitioned}.
  \end{center}
\end{frame}

\begin{frame}{Correct solution}
  \begin{center}
    In mathematical notation, these two conditions can be represented as\\
    \quad
    \begin{alignat}{3}
      \bigcup P = X \quad \quad \quad \quad\\
      A \cap B = \varnothing \quad \quad
      \forall 
      \begin{cases}
        A \subset X\\
        B \subset X\\
        A \neq B\\
      \end{cases}
    \end{alignat}
  \end{center}
\end{frame}

\begin{frame}{Optimal solution}
  \begin{center}
    Optimal solution can be obtained by minimizing, the following expression.\\
    \quad
    \begin{alignat}{3}
      D ( A_1, A_2, ... A_n) = max \left \{ \sum_{x \in A_i} x \right \} - min \left \{ \sum_{x \in A_j} x \right \}
    \end{alignat}\\
    \quad \\
    Its ideal when global minima for this function is reached.
  \end{center}
\end{frame}

\begin{frame}{But set-partioning is an...}
  \begin{center}
    {\huge NP-complete problem}\\
    \quad \\
    \pause
    \quad \\
    Finding a polynomial time solution is not possible unless \text{P = NP}
  \end{center}
\end{frame}

\begin{frame}{Applications in real world}
  \begin{centering}
    \begin{itemize}
    \item Scheduling \text{X} tasks on \text{K} identicial processors
      \pause
    \item Airline crew scheduling
      \pause
    \item Truck delivery management etc.
    \end{itemize}
  \end{centering}
\end{frame}

\subsection{Common solutions...}

\begin{frame}{Deterministic approach}{Greedy algorithm}
  \begin{centering}
    \begin{itemize}
    \item Sort U(elements) in decending order of weights
      \pause
    \item Sort B(bags) in ascending order
      \pause
    \item Remove the first/maximum-value element from U
      \pause
    \item Add P it to the first/lowest-weight bag B
      \pause

    \end{itemize}
  \end{centering}
\end{frame}

\begin{frame}{Deterministic approach}{Differencing algorithm}
  \begin{centering}
    \begin{itemize}
    \item Select $$x_i\quad and\quad x_j\quad :\quad | x_i - x_j |\quad where\quad  x_n \in U$$
      \pause
    \item Ensure $$x_i\quad and\quad x_j$$ go in different bags
      \pause
    \item Reinsert the difference back in U
      \pause
    \item Repeat from the first step until $$U = \varnothing$$
      \pause
    \item Read the allocation backwards while subtracting the inserted members to get fair sets
    \end{itemize}
    \pause
  \end{centering}
\end{frame}

\section{Genetic Algorithm}

\subsection{Methodology}

\begin{frame}{Representation}
  \begin{itemize}
  \item Every bag/set is a single parent chromosome strand
    \pause
  \item It can be thought of as a bit string
  \end{itemize}
  \pause
  For instance, if
  \begin{alignat}{3}
    U = \left \{ x_o, x_1, x_2 ... x_{n-1} \right \} 
  \end{alignat}
  And 0...n-1 being indexes of bitstring
  The act of selecting or rejecting becomes $$ x_i \in \left \{\quad 0,\quad 1\quad\right \}$$
  Where 1 means selection and 0 means rejection
\end{frame}

\begin{frame}{Representation}
  While the result of selecting a given sub-strand is uniquely identified, this can be simplified to be
  \pause
  \begin{itemize}
  \item A list elements the belong to 
    \begin{alignat}{3}
      A_i\quad 
      \begin{cases}
        \quad 1\quad <\quad i\quad <\quad k
      \end{cases}
    \end{alignat}
    \pause
  \item A cross-over can hence be represented as exchanging
    \begin{alignat}{3}
      x_{i}...x_{i + \delta}\quad and\quad y_{j}...y_{j + \delta'}
      \begin{cases}
        x \in Parent A_x\\
        y \in Parent A_y\\
        while\quad A_x\quad and\quad A_y\quad are parents
      \end{cases}
    \end{alignat}
  \end{itemize}
\end{frame}

\begin{frame}{Fitness Fn}{means of improvement}
  \begin{center}
    The new generation
    \begin{alignat}{2}
      A_x' =x_1...x_{i} + y_j...y_{j + \delta'} + x_{j+\delta+1}...x_n\\
      A_y' =y_1...y_{j} + x_i...x_{i + \delta} + y_{j+\delta'+1}...y_m
    \end{alignat}
    Can be considered better than parents $$A_x\quad and\quad A_y\quad iff:$$
    \begin{alignat}{2}
      SD(A_x,A_y) > SD(A_x', A_y') 
      \begin{cases}
        \bar{A} = \frac{\sum_{i=1}^XU_{x_i}}{k}\quad :\quad k = no.\quad of\quad bags
      \end{cases}
    \end{alignat}
    This is just one way to mesure the evolution of the new generation.\\
  \end{center}
\end{frame}

\begin{frame}{Termination}{No of generations}
  \begin{center}
    {\large First cut of the algorithm terminates after a pre-defined number of iterations}\\
    \quad\\
    {\small We are studying it further to build huristics to terminate it more aggresively }
  \end{center}
\end{frame}

\subsection{Experiments \& Result}

\begin{frame}{Random numbers}{0 < n < 100; 10 generations}
  \begin{centering}
    \pgfimage[width=11cm]{images/under-100-random-nos-10.png}
  \end{centering}
\end{frame}

\begin{frame}{Random numbers}{0 < n < 100; 100 generations}
  \begin{centering}
    \pgfimage[width=11cm]{images/under-100-random-nos-100.png}
  \end{centering}
\end{frame}

\begin{frame}{Random numbers}{0 < n < 100; 1000 generations}
  \begin{centering}
    \pgfimage[width=11cm]{images/under-100-random-nos-1000.png}
  \end{centering}
\end{frame}

\begin{frame}{Random numbers}{0 < n < 10000; 10 generations}
  \begin{centering}
    \pgfimage[width=11cm]{images/under-10000-random-nos-10.png}
  \end{centering}
\end{frame}

\begin{frame}{Random numbers}{0 < n < 10000; 100 generations}
  \begin{centering}
    \pgfimage[width=11cm]{images/under-10000-random-nos-100.png}
  \end{centering}
\end{frame}

\begin{frame}{Random numbers}{0 < n < 10000; 1000 generations}
  \begin{centering}
    \pgfimage[width=11cm]{images/under-10000-random-nos-1000.png}
  \end{centering}
\end{frame}

\begin{frame}{Random numbers}{0 < n < 10000; 10000 generations}
  \begin{centering}
    \pgfimage[width=11cm]{images/under-10000-random-nos-10000.png}
  \end{centering}
\end{frame}

\begin{frame}{Band distribution of numbers}{0 < 60\%X'es < 20\%; 60\% < 20\%X'es < 80\%; 80\% < 20\%X'es; 10000 generations}
  \begin{centering}
    \pgfimage[width=11cm]{images/band-nos-10000-iter.png}
  \end{centering}
\end{frame}

\subsection{Future direction}

\begin{frame}{More deterministic crossover}
  \begin{centering}
    \begin{itemize}
      \item First few crossovers can be made more deterministic
        \pause
      \item Until it becomes harder to do deterministic crossovers
        \pause
      \item Match inverted bell curve to bring higher-parent down to mean
        \pause
      \item At the same time, bring lower-parent up to mean
    \end{itemize}
  \end{centering}
\end{frame}

\begin{frame}{Parallel evolution}
  \begin{centering}
    \begin{itemize}
    \item Multicore machines are now cheep
      \pause
    \item Run generations in-parallel
      \pause
    \item Have barriers to sync, carry forward only best few
      \pause
    \item Say 50\% of not-so-evolved generations get dropped
      \pause
    \item The other 50\% get double up to make up for the dropped ones, and continue evolving
    \end{itemize}
  \end{centering}
\end{frame}

\begin{frame}{Other things to try}
  \begin{centering}
    \begin{itemize}
    \item Find benchmark data, run with it
      \pause
    \item Compare with approximation algorithms(greedy, differential)
      \pause
    \item Permute/Bruteforce to get ideal partitioning, compare with it
      \pause
    \item Get test-case time data, see how it specifically behaves for real world test-times distribution
      \pause
    \item Integrate with TLB, see how it performs on the floor
    \end{itemize}
  \end{centering}
\end{frame}

\section*{Thank you}

\begin{frame}{Thank you}
  \begin{centering}
    \quad\\
        {\huge Thank you.}\\
        \quad\\
        \quad\\
        Special thanks to Dr. H K Anasuya Devi(CCE, IISc Bangalore) and Asst.Prof. Satish Govindrajan(CSA, IISc Bangalore) for motivation and guidence.
        \quad\\
        \quad\\
        References:\\
        http://github.com/test-load-balancer/set-part\\
  \end{centering}
\end{frame}

% All of the following is optional and typically not needed. 
%% \appendix
%% \section<presentation>*{\appendixname}
%% \subsection<presentation>*{For Further Reading}

%% \begin{frame}[allowframebreaks]
%%   \frametitle<presentation>{For Further Reading}

%%   \begin{thebibliography}{10}

%%   \beamertemplatebookbibitems
%%   % Start with overview books.

%%   \bibitem{Author1990}
%%     A.~Author.
%%     \newblock {\em Handbook of Everything}.
%%     \newblock Some Press, 1990.


%%   \beamertemplatearticlebibitems
%%   % Followed by interesting articles. Keep the list short. 

%%   \bibitem{Someone2000}
%%     S.~Someone.
%%     \newblock On this and that.
%%     \newblock {\em Journal of This and That}, 2(1):50--100,
%%     2000.
%%   \end{thebibliography}
%% \end{frame}

\end{document}
