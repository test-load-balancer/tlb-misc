\documentclass{beamer}
\mode<presentation>
{
  \usetheme{Warsaw}
  \setbeamercovered{transparent}
}


\usepackage[english]{babel}
\usepackage[latin1]{inputenc}
\usepackage{times}
\usepackage[T1]{fontenc}

\title[TLB - https://github.com/test-load-balancer]
      {TLB}

\subtitle
    {a test load balancer for you}

\author[Pavan, Janmejay]
       {Pavan KS\inst{1} \and Janmejay Singh\inst{2}}

\institute[ThoughtWorks Studios]
{
  \inst{1}%
  mail: itspanzi@gmail.com\\
  blog: http://itspanzi.blogspot.com
  \and
  \inst{2}%
  mail: singh.janmejay@gmail.com\\
  blog: http://codehunk.wordpress.com
}

\date[xconf] 
{XConf Pune, 2010}

\subject{Introduction to internals of TLB, and its usage}
% \pgfdeclareimage[height=0.5cm]{logo}{logo-filename}
% \logo{\pgfuseimage{logo}}

\AtBeginSubsection[]
{
  \begin{frame}<beamer>{Outline}
    \tableofcontents[currentsection,currentsubsection]
  \end{frame}
}

% If you wish to uncover everything in a step-wise fashion, uncomment the following command: 
%\beamerdefaultoverlayspecification{<+->}

\begin{document}

\begin{frame}
  \titlepage
\end{frame}

\begin{frame}{Outline}
  \tableofcontents
  % You might wish to add the option [pausesections]
\end{frame}

\section{Motivation}

\subsection{The basic problem that we solved}

\begin{frame}{This is the story of how we went from...}
  \begin{centering}
    \pgfimage[width=10cm]{images/no-balance-go.png}
    \par
  \end{centering}
\end{frame}

\begin{frame}{to...}
  \begin{centering}
    \pgfimage[width=10cm]{images/count-balance-go.png}
    \par
  \end{centering}
\end{frame}

\begin{frame}{to...}
  \begin{centering}
    \pgfimage[width=10cm]{images/time-balance-go.png}
    \par
  \end{centering}
\end{frame}

\begin{frame}{by changing}
  \begin{centering}
    ...just a few lines in the build script
  \end{centering}
\end{frame}

\subsection{Effect: Productivity Enhancement}

\begin{frame}
  \begin{centering}
    \begin{itemize}
    \item Fast Build results in Rapid Development
      \pause
    \item Devs spend less time waiting to checkin
      \begin{itemize}
      \item Need not be limited to unit/integration tests          
      \item Functional/Acceptance tests
      \end{itemize}
      \pause
    \item Easier for devs to run precommit builds - Results in pulling upstream changes often and running builds frequently
    \end{itemize}
  \end{centering}
\end{frame}


\begin{frame}{Some title}
  You can create overlays\dots
  \begin{itemize}
  \item using the \texttt{pause} command:
    \begin{itemize}
    \item
      First item.
      \pause
    \item    
      Second item.
    \end{itemize}
  \item
    using overlay specifications:
    \begin{itemize}
    \item<3->
      First item.
    \item<4->
      Second item.
    \end{itemize}
  \item
    using the general \texttt{uncover} command:
    \begin{itemize}
      \uncover<5->{\item
        First item.}
      \uncover<6->{\item
        Second item.}
    \end{itemize}
  \end{itemize}
\end{frame}


\subsection{Previous Work}

\begin{frame}{Make Titles Informative.}
\end{frame}

\begin{frame}{Make Titles Informative.}
\end{frame}



\section{Our Results/Contribution}

\subsection{Main Results}

\begin{frame}{Make Titles Informative.}
\end{frame}

\begin{frame}{Make Titles Informative.}
\end{frame}

\begin{frame}{Make Titles Informative.}
\end{frame}


\subsection{Basic Ideas for Proofs/Implementation}

\begin{frame}{Make Titles Informative.}
\end{frame}

\begin{frame}{Make Titles Informative.}
\end{frame}

\begin{frame}{Make Titles Informative.}
\end{frame}



\section*{Summary}

\begin{frame}{Summary}

  % Keep the summary *very short*.
  \begin{itemize}
  \item
    The \alert{first main message} of your talk in one or two lines.
  \item
    The \alert{second main message} of your talk in one or two lines.
  \item
    Perhaps a \alert{third message}, but not more than that.
  \end{itemize}
  
  % The following outlook is optional.
  \vskip0pt plus.5fill
  \begin{itemize}
  \item
    Outlook
    \begin{itemize}
    \item
      Something you haven't solved.
    \item
      Something else you haven't solved.
    \end{itemize}
  \end{itemize}
\end{frame}



% All of the following is optional and typically not needed. 
\appendix
\section<presentation>*{\appendixname}
\subsection<presentation>*{For Further Reading}

\begin{frame}[allowframebreaks]
  \frametitle<presentation>{For Further Reading}
    
  \begin{thebibliography}{10}
    
  \beamertemplatebookbibitems
  % Start with overview books.

  \bibitem{Author1990}
    A.~Author.
    \newblock {\em Handbook of Everything}.
    \newblock Some Press, 1990.
 
    
  \beamertemplatearticlebibitems
  % Followed by interesting articles. Keep the list short. 

  \bibitem{Someone2000}
    S.~Someone.
    \newblock On this and that.
    \newblock {\em Journal of This and That}, 2(1):50--100,
    2000.
  \end{thebibliography}
\end{frame}

\end{document}


