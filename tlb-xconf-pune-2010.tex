\documentclass{beamer}
\mode<presentation>
{
  \usetheme{Warsaw}
  \setbeamercovered{transparent}
}


\usepackage[english]{babel}
\usepackage[latin1]{inputenc}
\usepackage{times}
\usepackage[T1]{fontenc}
\usepackage{multicol}

\title[TLB - https://github.com/test-load-balancer]
      {TLB}

\subtitle
    {a test load balancer for you}

\author[Pavan, Janmejay]
       {Pavan KS\inst{1} \and Janmejay Singh\inst{2}}

\institute[ThoughtWorks Studios]
{
  \inst{1}%
  mail: itspanzi@gmail.com\\
  blog: http://itspanzi.blogspot.com
  \and
  \inst{2}%
  mail: singh.janmejay@gmail.com\\
  blog: http://codehunk.wordpress.com
}

\date[xconf] 
{XConf Pune, 2010}

\subject{Introduction to internals of TLB, and its usage}

\AtBeginSubsection[]
{
  \begin{frame}<beamer>{Outline}
    \tableofcontents[currentsection,currentsubsection]
  \end{frame}
}

% If you wish to uncover everything in a step-wise fashion, uncomment the following command: 
%\beamerdefaultoverlayspecification{<+->}

\begin{document}

\begin{frame}
  \titlepage
\end{frame}

\begin{frame}{Whats in it for me?}
  \begin{centering}
     {\large What can I expect out of this session?}\\
     \quad\\
    \begin{itemize}
      \item A small, hopefully interesting, story that describes \emph{a problem}
      \item How we solved \emph{that problem}
      \item How we can help you solve it
      \item How you can help us, help others solve it!
    \end{itemize}
  \end{centering}
\end{frame}

\begin{frame}{Outline}
  \tableofcontents
  % You might wish to add the option [pausesections]
\end{frame}

\section{Motivation}

\subsection{Problem that we solved}

\begin{frame}
  \begin{center}
    {\huge The Story}
  \end{center}
\end{frame}

\begin{frame}{This is the story of how we went from...}
  \begin{centering}
    \pgfimage[width=10cm]{images/no-balance-go.png}
    \par
  \end{centering}
\end{frame}

\begin{frame}{to...}
  \begin{centering}
    \pgfimage[width=10cm]{images/count-balance-go.png}
    \par
  \end{centering}
\end{frame}

\begin{frame}{to...}
  \begin{centering}
    \pgfimage[width=10cm]{images/time-balance-go.png}
    \par
  \end{centering}
\end{frame}

\begin{frame}{by changing}
  \begin{centering}
    ...just a few lines in the build script
  \end{centering}
\end{frame}

\subsection{Dream: Fast Builds}

\begin{frame}
  \begin{centering}
    \begin{itemize}
    \item Fast build = Rapid development
      \pause
    \item Devs spend less time waiting to checkin
      \begin{itemize}
      \item Need not be limited to unit/integration tests          
      \item Functional/Acceptance tests
      \end{itemize}
      \pause
    \item Easier for devs to run precommit builds - Results in pulling upstream changes often and running builds frequently
    \end{itemize}
  \end{centering}
\end{frame}

\begin{frame}{Also we know that...}
  \begin{centering}
    \begin{itemize}
    \item Major part of build time is spent in running tests
      \pause
    \item Speeding up builds is non trivial
      \pause
    \item Most TW teams have a dev task for this don\text{'}t they??
    \end{itemize}
  \end{centering}
\end{frame}

\begin{frame}{Common solutions...}
  \begin{centering}
    \begin{itemize}
    \item Split applications into modules
      \pause
      \begin{itemize}
      \item Difficult to model (typically end up in diamond dependencies)
        \pause
      \item If downstream dependencies fail, turn around time to fix is huge
      \end{itemize}      
      \pause
    \item Throw more hardware at it - Slice and dice
      \pause
      \begin{itemize}
      \item Hand written partitioning using directories/tags etc (unequal partitions)
        \pause
      \item Pipelines (unit -> integration -> smoke -> functional) (serial process)
      \end{itemize}
    \end{itemize}
    \pause
  \end{centering}
  \pause
  Logical! but Suboptimal :-(\\
  \pause
  Nither too efficient nor effective.
\end{frame}

\section{TLB}

\subsection{Introducing TLB}

\begin{frame}{What if partitioning can be off-loaded?}
  What does TLB do?
  \begin{itemize}
  \item Makes \texttt{n} partitions
    \pause
  \item Understands which partition the \emph{current test runner process} is
    \pause
  \item Each partition runs only \texttt{one} of the \texttt{n} \emph{mutually exclusive \& collectively exhaustive} sets
  \end{itemize}
\end{frame}

\begin{frame}{For example:}
  Some potential balancing strategies could be
  \pause
  \begin{itemize}
    \item Partition tests to make every set have equal number of tests
      \pause
    \item Or to have every set take about the same time to finish
  \end{itemize}
  \pause
  Some of these strategies require a central place to store and retrive test-data(running-time, result etc)
\end{frame}

\begin{frame}{Setup}{Typical Setup}
  \begin{centering}
    \pgfimage[width=9.5cm]{images/typical-setup.png}
  \end{centering}
\end{frame}

\begin{frame}{Setup}{Tlb Server}
  \begin{centering}
    \pgfimage[width=9.5cm]{images/server-focused.png}
  \end{centering}
\end{frame}

\begin{frame}{Setup}{Tlb Client}
  \begin{centering}
    \pgfimage[width=9.5cm]{images/client-focused.png}
  \end{centering}
\end{frame}

\begin{frame}{Setup}{Server-Client Communication}
  \begin{centering}
    \pgfimage[width=9.5cm]{images/server-client-communication-focused.png}
  \end{centering}
\end{frame}

\begin{frame}{Setup}{Test Runner}
  \begin{centering}
    \pgfimage[width=9.5cm]{images/runner-focused.png}
  \end{centering}
\end{frame}

\begin{frame}{Setup}{Client-Runner Communication}
  \begin{centering}
    \pgfimage[width=9.5cm]{images/client-runner-communication-focused.png}
  \end{centering}
\end{frame}

\begin{frame}{Setup}{Alien Environment Setup}
  \begin{centering}
    \pgfimage[width=9.5cm]{images/alien-env-setup.png}
  \end{centering}
\end{frame}

\begin{frame}{Communication}{Server-Client-Runner Talk}
  \begin{multicols}{2}
    \begin{centering}
      \pgfimage[height=6cm]{images/server-client-timeline-1.png}
    \end{centering}
    \begin{center}
      {\huge Step 1}\\
      \quad\\
          {\large Receive list of to-be-run tests from the testing-framework}
    \end{center}
  \end{multicols}
\end{frame}

\begin{frame}{Communication}{Server-Client-Runner Talk}
  \begin{multicols}{2}
    \begin{centering}
      \pgfimage[height=6cm]{images/server-client-timeline-2.png}
    \end{centering}
    \begin{center}
      {\huge Step 2}\\
      \quad\\
          {\large Fetch historical test data from TLB server(tests that failed in the previous run/time taken by each test)}
    \end{center}
  \end{multicols}
\end{frame}

\begin{frame}{Communication}{Server-Client-Runner Talk}
  \begin{multicols}{2}
    \begin{centering}
      \pgfimage[height=6cm]{images/server-client-timeline-3.png}
    \end{centering}
    \begin{center}
      {\huge Step 3}\\
      \begin{itemize}
          \item {\large Prune the list of to-be-run tests to get tests to be actually executed(other partitions take care of pruned items)}
          \item {\large Re-order the pruned list,for instance \emph{pull tests that failed in the previous run to execute first}}
      \end{itemize}
    \end{center}
  \end{multicols}
\end{frame}

\begin{frame}{Communication}{Server-Client-Runner Talk}
  \begin{multicols}{2}
    \begin{centering}
      \pgfimage[height=6cm]{images/server-client-timeline-4.png}
    \end{centering}
    \begin{center}
      {\huge Step 4}\\
          \begin{itemize}
          \item {\large Let the party begin, \emph{execute the pruned tests}}
          \item {\large Continue capturing test-result/test-time as the suites run}
          \end{itemize}
    \end{center}
  \end{multicols}
\end{frame}

\begin{frame}{Communication}{Server-Client-Runner Talk}
  \begin{multicols}{2}
    \begin{centering}
      \pgfimage[height=6cm]{images/server-client-timeline-5.png}
    \end{centering}
    \begin{center}
      {\huge Step 5}\\
      \quad\\
          {\large Some feedback to the server; post test-run-time/test-results back, seeding data for future runs}
    \end{center}
  \end{multicols}
\end{frame}


\begin{frame}{Communication}{Server-Client-Runner Talk}
  \begin{multicols}{2}
    \begin{centering}
      \pgfimage[height=6cm]{images/server-client-timeline-6.png}
    \end{centering}
    \begin{center}
      {\huge Step 6}\\
      \quad\\
          {\large Terminate gracefully; build task returns}
    \end{center}
  \end{multicols}
\end{frame}

\begin{frame}{Ok enough of handwaving!}
  \begin{center}
    Well, that was all too abstract, lets see what TLB has to offer in terms of concrete features.
  \end{center}
\end{frame}

\subsection{Concepts in TLB}

\begin{frame}{TLB Client}{the workhorse}
  TLB client has two major sub-units.
  \pause
  \quad\\
  \begin{itemize}
  \item Balancer - \emph{the pruner guy} ({\small chosen by setting an environment variable \text{TLB\_CRITERIA}})
    \quad\\
    \pause
  \item Orderer - \emph{the shuffler guy} ({\small chosen by setting an environment variable \text{TLB\_ORDERER}})
  \end{itemize}
  {\tiny both environment variables require fully qualified java class-names}
\end{frame}

\begin{frame}{TLB Client}{Balancer}
  \begin{center}
    {\large Count-based Balancing}\\
    \quad\\
    \quad\\
    20 tests / 4 splits = 5 tests on each\\
    \quad\\
    \quad\\
    \begin{itemize}
    \item Conceptually straight-forward
    \item Inefficient in practice
    \item TLB uses this as a fallback, not recomended as preffered algorithm
    \end{itemize}
  \end{center}
\end{frame}

\begin{frame}{TLB Client}{Balancer}
  \begin{center}
    {\large Time-based Balancing}\\
    {\tiny (inspired by Amdahl`s law)}\\
    \quad\\
    \quad\\
    N tests / 4 splits $\approx$ 4 splits that take equal time
    \quad\\
    \quad\\
    \begin{itemize}
    \item Much better, yields fairly close to ideal solution
    \item One slow machine can not only slow down the current run, but skew balancing on the next one too
    \end{itemize}
  \end{center}
\end{frame}

\begin{frame}{TLB Client}{Balancer}
  \begin{center}
    {\large Smoothened Time-based Balancing}\\
    {\tiny (Ensures no outliers, builds on top of time based balancing)}\\
    \quad\\
    {\small N tests / 4 splits $\approx$ 4 splits which take equal time based on history over past few/several runs \\
    While exponential smoothing, every test-time entry $ S_t \quad \forall t > 0 $ is recorded as:}
    \begin{alignat}{3}
      S_1 & = x_o\\
      S_t & = \alpha x_{t-1} + (1 - \alpha)S_{t-1} \quad
      \forall 
      \begin{cases}
        t > 1\\ 
        0 < \alpha < 1
      \end{cases}
    \end{alignat}
    {\tiny Where $\alpha$ is the factor of smoothing, which can be tuned externally and \texttt{x} is unsmoothed reading.}
  \end{center}
\end{frame}

\begin{frame}{TLB Client}{Balancer}
  \begin{center}
    {\large Default-chain Balancing}\\
    \quad\\
    \begin{itemize}
    \item Allows users to define \emph{criteria chain}, which is a COLON(\text{:}) seperated list of algorithms
    \item The chain may include some custom balancer recepies of your own
    \item Used to ensure build doesn't fail when no data available to do advanced algorithms like Time-balance
    \item Allows defaulting to simpler algorithms like Count-balancing
    \end{itemize}
  \end{center}
\end{frame}

\begin{frame}{TLB Client}{Balancer}
  \begin{center}
    {\large *XYZ* Balancing}\\
    {\tiny (This is not a canned algorithm, its something you can create)}\\
    \quad\\
    \begin{itemize}
    \item You can create your custom balancing algorithm, and use it with TLB
    \item The contract is enforced by a java \text{abstract class} called \emph{TestSplitterCriteria}
    \item If it can potentially fail in some situations and you want a fallback, you can use \emph{DefaultingTestSplitterCriteria} with your balancer in chain
    \item Note: Algorithm need to be repeatable, since its executed on every partition. \emph{Mutual-exclusion \& Collective-exhausion} are imparative.
    \end{itemize}
  \end{center}
\end{frame}

\begin{frame}{TLB Client}{Orderer}
  \begin{center}
    {\large Failed First Orderer}\\
    {\tiny (Runs tests that failed last time around, first)}\\
    \begin{itemize}
      \item Perfect for fixing builds that have a tendency to break after 6 in the evening
      \item You don\text{'}t need to wait for the entire build, just watch the console log for a few minutes, as you see the test you fixed pass and scroll by
    \end{itemize}
  \end{center}
\end{frame}

\begin{frame}{TLB Client}{Orderer}
  \begin{center}
    {\large *ABC* Orderer}\\
    {\tiny (This is not a canned algorithm, its something you can create)}\\
    \quad\\
    \begin{itemize}
    \item You can create your custom ordering algorithm, and use it with TLB
    \item The contract is enforced by a java \text{abstract class} called \emph{TestOrderer}
    \item Ordering tests to ensure execution-order/side-effects is a slippery slope and is considered an \emph{ANTI PATTERN}, so we strongly recomend not abusing Ordering facility
    \end{itemize}
  \end{center}
\end{frame}

\subsection{Show me the code honey!}

\begin{frame}
  \begin{centering}
    {\small TLB Client needs to embed in build environment and interact with a testing-framework}
    \quad\\
    \quad\\
    \pause
    \begin{itemize}
    \item Balancing JUnit test-suite using Apache-Ant
      \pause
    \item Balancing RSpec test-suite using Rake
    \pause
   \item Using TLB on the Sears-Commercial RSpec suite \\
     {\small Thanks Jaju, for taking all the hell we gave you! :-)}
     \begin{itemize}
     \item {\small In all modesty, it took us half-an-hour to setup once we could build locally, the searial way. ;-)}
     \end{itemize}
    \end{itemize}
  \end{centering}
\end{frame}

\subsection{Hooking TLB up with your build process}

\begin{frame}{Leverage parallel execution capabilities of}{tools like...}
  \begin{center}
    \pgfimage[height=1.5cm]{images/go-logo.png}
  \end{center}
\end{frame}

\begin{frame}{or for that matter}
  \begin{center}
    {\huge Hudson, Bamboo, TeamCity, Ant Hill Pro}(if you are rich enough),{\huge or even Capistrano/Shell script fork}(if you are a poor dev like us).\\
    \quad\\\quad\\
        {\footnotesize PS: You did not see this slide :-)}
  \end{center}
\end{frame}

\begin{frame}{Supported Frameworks}
  \begin{centering}
    \begin{itemize}
      \item {\color{green}Junit} using {\color{red}Ant} or {\color{red}Buildr} for {\color{blue}Java}
        \pause
      \item {\color{green}Twist} using {\color{red}Ant} or {\color{red}Buildr} for {\color{blue}Java}
        \pause
      \item {\color{green}Rspec} using {\color{red}Rake} for both {\color{blue}MRI} and {\color{blue}JRuby}
    \end{itemize}
  \end{centering}
\end{frame}

\begin{frame}{We plan to support...}
  \begin{centering}
    \begin{multicols}{2}
      \begin{itemize}
      \item {\color{green}NUnit} on {\color{blue}.Net}
      \item {\color{green}MS Test} on {\color{blue}.Net}
      \item {\color{green}TestUnit} on {\color{blue}MRI \& JRuby}
      \item {\color{green}Cucumber} on {\color{blue}MRI \& JRuby}
      \item {\color{green}TestNG} on {\color{blue}Java}
      \item {\color{green}PyUnit} on {\color{blue}Python}
      \item {\color{green}CPPUnit} on {\color{blue} C++}
      \item {\color{green}5am (fiveam)} on {\color{blue} CommonLisp}
      \end{itemize}
      \begin{itemize}
      \item {\color{red}NAnt} on {\color{blue}.Net}
      \item {\color{red}MS Build} on {\color{blue}.Net}
      \item {\color{red}Maven} on {\color{blue}Java}
      \end{itemize}
    \end{multicols}
  \end{centering}
\end{frame}

\begin{frame}{While thats our wish-list}
  \begin{center}
    \begin{itemize}
    \item Bad news is, we haven't started work on most of these yet.\\
      \pause
    \item Good news is, we have good hackers, like yourself, listening to us here, who can help!
    \end{itemize}
    \pause
    \quad\\
    We'd love to support anything else that you can make time to contribute :-)
  \end{center}
\end{frame}

\section{Dev Adrenaline}

\subsection{Polynomial Time Set-Partitioning}

\begin{frame}{Partitioning Approaches}
  \begin{itemize}
    \item Greedy Algorithm (time-balancing-criteria uses this one)
    \item Differencing Algorithm (half done, not upstrem yet)
    \item GA spike, terrifingly good (upstream as a spike, written in CommonLisp, needs some more tuning)
  \end{itemize}
\end{frame}

\begin{frame}{GA spike for set-partitioning}{TLB doesn't have it yet; coming soon}
  \begin{center}
    \pgfimage[width=11cm]{images/ga-shot.png}
  \end{center}
\end{frame}

\subsection{Freestyle}

\begin{frame}{Freestyle}
  On a freestyle day, you get to work on a solving a problem of your choice, related to the product/process/team.
  \begin{itemize}
    \item We try to do it \emph{one day per fort-night}
    \item It bootstrapped TLB
    \item Even though probably less than 10\% of TLB was written in Freestyle time, it was a necessary catalyst
  \end{itemize}
\end{frame}

\section*{Thank you}

\begin{frame}{Thank you}
  \begin{centering}
  {\huge We are patch hungry*.\\Please Contribute.\\}
  (its BSD 2 clause)\\
  \quad\\
  {\huge Thank you.}\\
  \quad\\
  \quad\\
  References:\\
  http://github.com/test-load-balancer/tlb\\
  http://github.com/test-load-balancer/tlb\_rb\\
  \quad\\
  * http://github.com/test-load-balancer/tlb/issues\\
  \end{centering}
\end{frame}

% All of the following is optional and typically not needed. 
%% \appendix
%% \section<presentation>*{\appendixname}
%% \subsection<presentation>*{For Further Reading}

%% \begin{frame}[allowframebreaks]
%%   \frametitle<presentation>{For Further Reading}
    
%%   \begin{thebibliography}{10}
    
%%   \beamertemplatebookbibitems
%%   % Start with overview books.

%%   \bibitem{Author1990}
%%     A.~Author.
%%     \newblock {\em Handbook of Everything}.
%%     \newblock Some Press, 1990.
 
    
%%   \beamertemplatearticlebibitems
%%   % Followed by interesting articles. Keep the list short. 

%%   \bibitem{Someone2000}
%%     S.~Someone.
%%     \newblock On this and that.
%%     \newblock {\em Journal of This and That}, 2(1):50--100,
%%     2000.
%%   \end{thebibliography}
%% \end{frame}

\end{document}
